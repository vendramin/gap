\chapter{Combinatorics and topology}

In this chapter we will see that some combinatoric structures are implemented in GAP, such as graphs, clique complexes, simplicial complexes (and more generally, regular CW-complexes) and finite posets.
We will briefly explain how to use some basic commands to work with these structures.
Some of them may not be included in the basic installation of GAP and so require installation.
Below we list some useful packages:

\begin{enumerate}
    \item GRAPE: this package contains an implementation of $G$-graphs, that is, graphs together with a group action.
    
    \item HAP: this GAP package allows us to construct regular CW-complexes or simplicial complexes. With it, we can compute their homology, cohomology and fundamental. It also implements chain complexes and some group cohomology.
    
    \item Posets: this is an implementation of finite posets and finite topological spaces, which may be useful to simplify some calculations or even construct simplicial complexes as order-complexes.
\end{enumerate}

{\color{red} Add references.}

We will give a very brief introduction to these packages. The reader can consult the documentation of these packages to get more detailed explanation of the commands available.

\section{Relations}

\begin{example}
Let $A=\{2,\dots,100\}$ and let $\sim$ be the binary
relation given by $n\sim m$ if and only if $\gcd(n,m)=1$.
We construct the binary relation
and then check whether the relation is
reflexive, (anti)symmetric and transitive.

\begin{lstlisting}
gap> A := [2..100];;
gap> l := [];;
gap> for x in A do
> for y in A do
> if Gcd(x,y)>1 then
> Add(l, Tuple([x,y]));
> fi;
> od;
> od;
gap> rel := BinaryRelationByElements(Domain(A), l);;
gap> IsReflexiveBinaryRelation(rel);
true
gap> IsSymmetricBinaryRelation(rel);
true
gap> IsAntisymmetricBinaryRelation(rel);
false
gap> 1IsTransitiveBinaryRelation(rel);
false
\end{lstlisting}

To finish, we want to compute the number
of elements that are related to $12$.

\begin{lstlisting}
gap> Number(A, x->Gcd(x,12)>1);
67
\end{lstlisting}

Can you compute the number of elements
related to 12 in the case
where $A=\{2,\dots,100000\}$?
\end{example}

\begin{example}
For $g$ a polynomial such that $g(0)\ne 0$
consider the sequence of polynomials
given by $f_1=Xg$ and
$f_{n+1}=(Xf'_n)^n$. Compute the multiplicity
of $0$ as a root of $f_n$.

Let us fix $g=1$ and construct the sequence up to
some bound, say only five terms.
\begin{lstlisting}
gap> N := 5;;
gap> x := Indeterminate(Rationals, "x");;
gap> g := 1;;
gap> f := [x*g];;
gap> for n in [2..N] do
> Add(f, (x*Derivative(f[n-1], x))^(n-1));
> od;
gap> f;
[ x, x, x^2, 8*x^6, 5308416*x^24 ]
\end{lstlisting}

One can make experiments and use different values of $g$. It is an exercise to
the reader to prove that the multiplicity of zero in $f_n$ is $n!$,
independently of $g$.
\end{example}


\section{Graphs}

We provide some examples of the use of the package \texttt{GRAPE} and its functionalities.
The background idea in this package is to construct graphs that admit the action of a given group $G$, and take advantage of this action to actually compute the graph with a more efficient algorithms.

For a group $G$, we can construct a $G$-graph with the command
\[ \text{\lstinline{Graph( G, L, act, rel )}}\]
where \lstinline{G} is a group, \lstinline{L} is a list of elements contained in a larger $G$-set with action given by the parameter \lstinline{act}, and \lstinline{rel} is a $G$-invariant boolean function.

The most basic example is the case of the trivial group $G = 1$.
We construct the graph on vertex set $[1,2,3,4,5]$ with relations $(i,i+1)$ for $1\leq i < 5$.

\begin{lstlisting}
gap> G := TrivialGroup();;
gap> L := [1,2,3,4,5];;
gap> act := function(x,y) return x; end;;
gap> rel := function(x,y) return AbsInt(x-y) = 1; end;;
gap> g:=Graph(G,L,act,rel);
rec(
  adjacencies := [ [ 2 ], [ 1, 3 ], [ 2, 4 ], [ 3, 5 ], [ 4 ]
     ], group := Group(()), isGraph := true,
  names := [ 1, 2, 3, 4, 5 ], order := 5,
  representatives := [ 1, 2, 3, 4, 5 ],
  schreierVector := [ -1, -2, -3, -4, -5 ] )
\end{lstlisting}
We can check which vertices are adjacent to a given one by calling \lstinline{Adjacency}:
\begin{lstlisting}
gap> Adjacency(g,1);
[ 2 ]
gap> Adjacency(g,3);
[ 2, 4 ]
\end{lstlisting}
We can check if some edge is contained in the graph:
\begin{lstlisting}
gap> IsEdge(g, [1,2]);
true
gap> IsEdge(g, [1,3]);
false
\end{lstlisting}

Now let's construct a graph with a non-trivial $G$ action.
We take a set $L$ which is a (not necessary $G$-invariant) subset of a larger $G$-set $S$.

For example, take $L$ to be a set of representatives of the conjugacy classes of elements of $G$.
We take the graph $g$ on the elements of $G$, and we join two elements if and only if they commute.
The following code of GAP produces such commuting-graph for the symmetric group on $6$ elements.
\begin{lstlisting}
gap> G:=SymmetricGroup(6);;
gap> L:=List(ConjugacyClasses(G),Representative);;
gap> act:=function(x,perm) return x^perm; end;;
gap> rel:=function(x,y) return x^y = x; end;;
gap> g:=Graph(G,L,act,rel);;
\end{lstlisting}
Now we can alternatively compute the centralizer of an element of $G$ by looking its adjacent vertices in this graph.

Here we have to be careful: the vertices in the graph $g$ are integers $[1..n]$ where $n$ is the order of the graph $g$.
To recover the name of the original vertex labelled with \lstinline{i}, we can call the function \lstinline{VertexName(g,i)}.

\begin{lstlisting}
gap> VertexName(g,2);
(1,2)
gap> List(Adjacency(g,2),j->VertexName(g,j));
[ (), (1,2), (3,4), (1,2)(3,4) ]
gap> Elements(Centralizer(G, (1,2)));
[ (), (3,4), (1,2), (1,2)(3,4) ]
\end{lstlisting}

Many properties of a graph can be checked by using the commands of the GRAPE package.
For example, we have available the following commands: \lstinline{IsEdge}, \lstinline{OrderGraph}, \lstinline{VertexDegree}, \lstinline{IsSimpleGraph},  \lstinline{Distance}, \lstinline{IsConnectedGraph},
 \lstinline{Diameter}, \lstinline{Girt},
 \lstinline{IsBipartite}, \lstinline{IsCompleteGraph}, \lstinline{IsRegularGraph}, etc. %\lstinline{IsDistanceRegular}, \lstinline{ChromaticNumber}, \lstinline{CliqueNumber}, 


It is also possible to construct a graph from an adjacency matrix, as the following command shows.

\begin{lstlisting}
gap> A:=[ [0,0,0,0,0], [0,0,1,1,0], [0,1,0,0,1], [0,1,0,0,1], [0,
0,1,1,0] ];;
gap> g:=Graph( TrivialGroup(), [1,2,3,4,5], function(x,y) return 
x; end, function(x,y) return A[x][y]=1; end);;
gap> IsConnectedGraph(g);
false
gap> ConnectedComponents(g);
[ [ 1 ], [ 2, 3, 4, 5 ] ]
\end{lstlisting}

We can also access to the connected component of a vertex with the command \lstinline{ConnectedComponent( g, v )}.

\begin{lstlisting}
gap> ConnectedComponent(g,2);
[ 2, 3, 4, 5 ]
\end{lstlisting}

The command \lstinline{IndependentSet} allows us to construct independent sets of a graph.
Note that this implements a greedy algorithm that not always returns a maximal independent set.

\begin{lstlisting}
gap> IndependentSet(g);
[ 1, 2, 5 ]
gap> IndependentSet(g, [1,3]);
[ 1, 3, 4 ]
\end{lstlisting}

We can also check if two graphs are isomorphic with \lstinline{IsIsomorphicGraph}.
Recall that the isomorphism problem of graphs is not known to be solvable in polynomial time nor to be NP-complete.

\begin{lstlisting}
gap> A:=[[0,0,0], [0,0,0], [0,0,0]];;
gap> B:=[[0,0,1,1], [0,0,1,1], [1,1,0,0], [1,1,0,0]];;
gap> C:=[[0,1,1,0], [1,0,0,1], [1,0,0,1], [0,1,1,0]];;
gap> G:=TrivialGroup();;
gap> rel:=function(x,y,M) return M[x][y] = 1; end;;
gap> relMatrix:=function(M)
>     local relM;
>       relM:=function(x,y)
>               return rel(x,y,M);
>       end;;
>       return relM;
> end;;
gap> act:=function(x,y) return x; end;;
gap> gA:=Graph(G,[1,2,3],act, relMatrix(A));;
gap> gB:=Graph(G,[1,2,3,4],act, relMatrix(B));;
gap> gC:=Graph(G,[1,2,3,4],act, relMatrix(C));;
gap> IsIsomorphicGraph(gA,gB);
false
gap> IsIsomorphicGraph(gB,gC);
true
\end{lstlisting}

If we also want to get a concrete isomorphism between two graphs, we can use the command \lstinline{GraphIsomorphism}.
This will return false if the graphs are not isomorphic.

\begin{lstlisting}
gap> GraphIsomorphism(gB,gC);
(2,4,3)
gap> GraphIsomorphism(gA,gB);
fail
\end{lstlisting}

Finally, the GRAPE package contains some inbuilt function to construct known graphs.
We give some examples below.

For the complete graph on $n$ vertices with associated permutation group $G$ we can use the command \lstinline{CompleteGraph( G, n )}.
We can omit $n$ and just call \lstinline{CompleteGraph(G)}, in which case $n$ is assumed to be largest point moved by $G$.

\begin{lstlisting}
gap> CompleteGraph(AlternatingGroup(4),4);
rec( adjacencies := [ [ 2, 3, 4 ] ],
  group := Alt( [ 1 .. 4 ] ), isGraph := true,
  isSimple := true, order := 4, representatives := [ 1 ],
  schreierVector := [ -1, 1, 1, 2 ] )
gap> CompleteGraph(AlternatingGroup(4),5);
rec( adjacencies := [ [ 2, 3, 4, 5 ], [ 1, 2, 3, 4 ] ],
  group := Alt( [ 1 .. 4 ] ), isGraph := true,
  isSimple := true, order := 5, representatives := [ 1, 5 ],
  schreierVector := [ -1, 1, 1, 2, -2 ] )
gap> CompleteGraph(SymmetricGroup(6));
rec( adjacencies := [ [ 2, 3, 4, 5, 6 ] ],
  group := Sym( [ 1 .. 6 ] ), isGraph := true,
  isSimple := true, order := 6, representatives := [ 1 ],
  schreierVector := [ -1, 1, 1, 1, 1, 1 ] )
\end{lstlisting}

For the Cayley graph of a group, we can call the function \lstinline{CayleyGraph(G, S)}, where $S$ is a subset of the group $G$ (which may be a generating set or not).
We can omit the set $S$ and call the function \lstinline{CayleyGraph(G)}, which takes as underlying set $S$ the set  \lstinline{GeneratorsOfGroup(G)}.

\begin{lstlisting}
gap> CayleyGraph(PSL(2,3), [1]);
rec( adjacencies := [ [  ] ],
  group := Group([ (1,2,3)(4,7,10)(5,9,11)(6,8,12), (1,4)(2,6)
      (3,5)(7,11)(8,10)(9,12) ]), isGraph := true,
  isSimple := true,
  names := [ (), (2,3,4), (2,4,3), (1,2)(3,4), (1,2,3),
      (1,2,4), (1,3,2), (1,3,4), (1,3)(2,4), (1,4,2),
      (1,4,3), (1,4)(2,3) ], order := 12,
  representatives := [ 1 ],
  schreierVector := [ -1, 1, 1, 2, 2, 2, 1, 1, 1, 1, 2, 1 ] )
gap> CayleyGraph(PSL(2,3));
rec( adjacencies := [ [ 2, 3, 4 ] ],
  group := Group([ (1,2,3)(4,7,10)(5,9,11)(6,8,12), (1,4)(2,6)
      (3,5)(7,11)(8,10)(9,12) ]), isGraph := true,
  isSimple := true,
  names := [ (), (2,3,4), (2,4,3), (1,2)(3,4), (1,2,3),
      (1,2,4), (1,3,2), (1,3,4), (1,3)(2,4), (1,4,2),
      (1,4,3), (1,4)(2,3) ], order := 12,
  representatives := [ 1 ],
  schreierVector := [ -1, 1, 1, 2, 2, 2, 1, 1, 1, 1, 2, 1 ] )
\end{lstlisting}

\section{Simplicial complexes and posets}

In this section, we will use the package \texttt{HAP} to work with simplicial complexes and computation of fundamental group and homology.

The package \texttt{simpcomp} also implements simplicial complexes, but here we will use the commands of \texttt{HAP}.

To construct a simplicial complex with \texttt{HAP}, we need to provide either the list of all simplices, or just the list of the maximal simplices.
The following commands show such functionality.

\begin{lstlisting}

\end{lstlisting}

Compute list of simplices of dimension...
Compute Euler Characteristic...
Compute (Co)Homology..
Coefficients
FundamentalGroup
Chain Complex, coefficients


\section{Group cohomology}

% https://www.gap-system.org/Packages/crime.html
%https://www.gap-system.org/Packages/cohomolo.html

\section{Problems}

\begin{prob}
Given the matrix group $\GL_3(4)$, compute its Cayley graph with respect to the set of elementary matrices.
\end{prob}

\begin{prob}
Compute the commuting graph of $\Sym_6$ on the set of its non-trivial $3$-subgroups.
\end{prob}

\begin{prob}
Let $G = \PSL_2(8)$.
\begin{enumerate}%[label=(\roman*)]
    \item Show that the commuting graph on the $2$-subgroups of $G$ is disconnected.
    \item Pick a connected component of this graph and show that the stabilizer of such component is a strongly $2$-embedded subgroup of $G$.
\end{enumerate}
\end{prob}

\begin{prob}
Let $G = \PSL_2(9)$.
\begin{enumerate}%[label=(\roman*)]
    \item Compute the poset of non-trivial solvable groups of $G$.
    \item Show that its fundamental group is free.
\end{enumerate}
\end{prob}

\begin{prob}
Let $V$ a $4$-dimensional vector space over $\mathbb{F}_3$.
Prove that the poset of non-trivial proper subspaces of $V$ is homotopy equivalent to a bouquet of $729$ spheres of dimension $2$.
\end{prob}

\begin{prob}
Let $G = \Sym_8$ and consider the set $S = \{ (1,2), (1,3), (1,4) \}$.
Construct the Cayley graph of $G$ on $S$ and show that the connected component of $1$ is isomorphic to the Cayley graph of $\Sym_4$ on the set $S$ (which is indeed isomorphic to the span of $G$).
\end{prob}
