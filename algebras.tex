\chapter{Non-commutative algebras}

\label{algebras}

\index{Fomin--Kirillov algebras}
\index{Knopper, J.}
\index{Cohen, A.}

Let $A$ be an algebra given by generators and relations. In this chapter we
address the following problems: 
\begin{itemize}
\item How can we check if $A$ is finite-dimensional? 
\item Can we compute the center or the radical of $A$? 
\end{itemize}
As well, we will also try to answer the following.
\begin{itemize}
	\item Let $t_1,\dots,t_n$ denote the generators of $A$. Can we check if two given expressions $a,a'$ in terms of the $t_i's$ are equal? In particular, can we decide if $a=0$?
	\item How can we deal with tensor products, i.e.~handle $A\otimes A$ or $A\otimes A'$ for another algebra $A'$? 
\end{itemize}


\section{Non-commutative Gr\"obner basis}

To address these problems our tool is the Gr\"obner basis package
\lstinline{gbnp}. The package was written by A. Cohen and J. Knopper. First we
need to load the package:
\begin{lstlisting}
gap> LoadPackage("gbnp");
\end{lstlisting}

\index{Fomin, S.}
\index{Kirillov, A.}
\index{Milisnki, }
\index{Schneider, H,--J.}
\index{Majid, S.}
\index{Fomin--Kirillov algebras}
The first two examples are related to Fomin--Kirillov algebras, a family of
quadratic algebras introduced independently by Fomin and
Kirillov~\cite{MR1667680}, Milinski and Schneider~\cite{MR1800714} and
Majid~\cite{MR2106930}. 

\begin{definition}
For $n\geq3$ let $\mathcal{E}_{n}$ be the algebra (over the rationals) with
generators $x_{(ij)}$, where $1\leq i<j\leq n$, and relations
\begin{align*}
	&x_{(ij)}^{2}=0,&\text{ for $1\leq i<j\leq n$,}\\
	&x_{(ij)}x_{(jk)}=x_{(jk)}x_{(ik)}+x_{(ik)}x_{(ij)},&\text{ for $1\leq i<j<k\leq
	n$,}\\
	&x_{(jk)}x_{(ij)}=x_{(ik)}x_{(jk)}+x_{(ij)}x_{(ik)},&\text{ for $1\leq i<j<k\leq n$,}\\
	&x_{(ij)}x_{(kl)}=x_{(kl)}x_{(ij)},&\text{ for any distinct $i,j,k,l$.}
\end{align*}
\end{definition}

\begin{example}
	\label{example:FK3}
	In this example we will construct the algebra $\mathcal{E}_3$ (over the
	rationals) with generators $a,b,c$ and relations
\begin{align*}
	a^2=0,&& 
	b^2=0,&&
	c^2=0,&&
	ab+ca+bc=0,&&
	ba+ac+cb=0.
 \end{align*}

%see~\cite{MR1667680}. 
Is this algebra finite-dimensional? We first construct the algebra and compute
a Gr\"obner basis of the ideals of relations:
\begin{lstlisting}
gap> A := FreeAssociativeAlgebraWithOne(Rationals,\
> "a", "b", "c");;
gap> a := A.a;;
gap> b := A.b;;
gap> c := A.c;;
gap> one := One(A);;
gap> rels := [ a^2, b^2, c^2, a*b+b*c+c*a, a*c+b*a+c*b ];;
gap> K := GP2NPList(rels);;
gap> G := SGrobner(K);;
\end{lstlisting}

We prove that $\dim\mathcal{E}_3=12$ and that 
\[
1,\;
a,\;
b ,\;
c ,\;
ab ,\;
ac ,\;
ba ,\;
bc ,\;
aba ,\;
abc ,\;
bac ,\;
abac
\]
is a linear basis of the algebra: 

\begin{lstlisting}
gap> DimQA(G, 3);
12
gap> basis := BaseQA(G, 3, [1, 1, 1]);;
gap> PrintNPList(basis);
 1 
 a 
 b 
 c 
 ab 
 ac 
 ba 
 bc 
 aba 
 abc 
 bac 
 abac 
\end{lstlisting}
%In order to be able to work with this finite-dimensional algebra it would be
%helpful to have its structure constants. The following code computes the
%structure constants table:
%
%\begin{lstlisting}
%tab := EmptySCTable(Sum(dim), 0);;
%for x in basis do
%  for y in basis do
%    l := [];
%    xy := MulQA(x, y, G);
%    if not IsZero(xy) then 
%      for k in [1..Size(xy[1])] do
%        pos := Position(List(basis, x->x[1][1]), xy[1][k]);
%        Add(l, [xy[2][k], pos]);
%      od;
%      SetEntrySCTable(tab, Position(basis, x), Position(basis, y), Flat(l));
%      PrintNP(xy); 
%    fi;
%  od;
%od;
%\end{lstlisting}
%
%Now we define the algebra given by these structure constants, computes its
%dimension and check that it is indeed an associative algebra. We also verify
%that the algebra is not semisimple: its radical is non-zero.
%
%\begin{lstlisting}
%gap> alg := AlgebraByStructureConstants(Rationals, tab);;
%gap> Dimension(alg);
%12
%gap> IsAssociative(alg);
%true
%gap> rad := RadicalOfAlgebra(alg);;
%gap> Dimension(rad);
%11
%\end{lstlisting}
\end{example}

\begin{remark}
	It is known that the algebras $\mathcal{E}_4$ and $\mathcal{E}_5$ are finite-dimensional, see
	Problems~\ref{problem:FK4} and~\ref{problem:FK5}. For $n>5$, this is not known. It is a conjecture that $\mathcal{E}_6$
	is infinite-dimensional; see also Problem~\ref{problem:FK6}.
\end{remark}

Recall that the Hilbert
series of a graded algebra $B=\oplus_{n\geq0}B_n$ is 
\[
H(t)=\sum_{n\geq0}\dim (B_n)t^n.
\]

\begin{example}
	Let $A$ be the algebra over $\F_2$ generated by $a,b,c,d$ with relations
	\begin{gather*}
		a^2=b^2=c^2=d^2=0,\\
		ba+db+ad=ca+bc+ab=da+cd+ac=cb+dc+bd=0,\\
		cad+bac+dab=0.
	\end{gather*}
	The following code proves that $\dim A=36$: 

\begin{lstlisting}
gap> A := FreeAssociativeAlgebraWithOne(GF(2), \
> "a", "b", "c", "d");;
gap> a := A.a;;
gap> b := A.b;;
gap> c := A.c;;
gap> d := A.d;;
gap> rels := [ a^2, b^2, c^2, d^2, \
> b*a+d*b+a*d, c*a+b*c+a*b, \
> d*a+c*d+a*c, c*b+d*c+b*d, \
> c*a*d+b*a*c+d*a*b ];;
gap> G := SGrobner(GP2NPList(rels));;
gap> DimQA(G,4);
36
\end{lstlisting}
\end{example}

\section{Structure constants}

Knowing the structure constants of an algebra makes some computer calculations
easier. 

\begin{example}
	\label{example:Clifford}
	For parameters $\alpha$ and $\beta$ let $C_{\alpha,\beta}$ be the algebra
	with generators $x,y$ and relations 
	\[
		x^2=\alpha,\quad
		y^2=\alpha,\quad
		xy+yx=\beta.
	\]
	Let us prove that the algebra $C_{\alpha,\beta}$ is $4$-dimensional with
	basis $1,x,y,xy$.  First we need to define the base field with two
	indeterminates \lstinline{a,b}:
\begin{lstlisting}
gap> field := FunctionField(Rationals, 2);;
gap> ind := IndeterminatesOfFunctionField(field);;
gap> a := ind[1];;
gap> b := ind[2];;
\end{lstlisting}
Now we define an associate algebra with generators \lstinline{x,y} over \lstinline{field}. 
\begin{lstlisting}
gap> A := FreeAssociativeAlgebraWithOne(field, "x", "y");;
gap> x := A.1;;
gap> y := A.2;;
gap> one := One(A);;
\end{lstlisting}
We put the defining relations in a list \lstinline{rels}, compute a
non-commutative Gr\"obner basis for the ideal of relations and compute the
dimension and a linear basis of the algebra with generators \lstinline{x,y} and
relations
\lstinline{rels}:
\begin{lstlisting}
gap> rels := [x^2-one*a, y^2-one*a, x*y+y*x-one*b];;
gap> G := SGrobner(GP2NPList(rels));;
gap> dim := DimQA(G,2);
4
gap> basis := BaseQA(G, 2, dim);;
gap> PrintNPList(basis);
 1 
 a 
 b 
 ab 
\end{lstlisting}

To make calculations we need to know the structure constants of the algebra.
The following code creates the table \lstinline{tab} of structure contants:
\begin{lstlisting}
gap> tab := EmptySCTable(dim, Zero(field));;
gap> for x1 in basis do
> for x2 in basis do
> l := [];
> xy := MulQA(x1, x2, G);
> if not IsZero(xy) then 
> for k in [1..Size(xy[1])] do
> pos := Position(List(basis, z->z[1][1]), xy[1][k]);
> Add(l, [xy[2][k], pos]);
> od;
> SetEntrySCTable(tab, Position(basis, x1), \
> Position(basis, x2), Flat(l));
> fi;
> od;
> od;
\end{lstlisting}

With this code we create a new object for our algebra, now given by its
structure contants.  With this object one can easily compute things such as the
center or the radical: 
\begin{lstlisting}
gap> alg := AlgebraByStructureConstants(field, tab);;
gap> Dimension(Center(alg));
1
gap> Dimension(RadicalOfAlgebra(alg));
0
\end{lstlisting}
\end{example}

\section{Another way for producing relations}

We start with an example, which uses the functions for dealing with variables and relations.
\begin{example}\label{example:Clifford_revisited}
Let us consider once agagin the algebra in Example 	\ref{example:Clifford}. Another way to produce this algebra is as follows.
\begin{lstlisting}
gap> GBNP.ConfigPrint("x","y");;
gap> a:=Indeterminate(Rationals,"a");;
gap> b:=Indeterminate(Rationals,"b");;
gap> u:=[[[]],[1]];;
gap> x:=[[[1]],[1]];;
gap> y:=[[[2]],[1]];;
gap> r1:=AddNP(MulNP(x,x),u,1,-a);;
gap> r2:=AddNP(MulNP(y,y),u,1,-a);  ;                    
gap> r3:=AddNP(AddNP(MulNP(x,y),MulNP(y,x),1,1),u,1,-b);;
gap> K:=[r1,r2,r3];;
\end{lstlisting}
In this example, let us have a closer look to the variables, relations and ideals, to find that they are made out of lists; and hence all the machinery we have learned in previous chapters applies. Indeed, we see that
\begin{lstlisting}
gap> x:=[[[1]],[1]];
[ [ [ 1 ] ], [ 1 ] ]
\end{lstlisting}
is a list of two lists, both with a single element: the second \lstinline{[ 1 ]} contains a coefficient, namely the number one, that multiplies the first component of the first list \lstinline{ [ [ 1 ] ]}, that contains, itself, a list \lstinline{ [ 1 ]} with a single element, the first variable we have declared, \lstinline{"x"}. We take a look to a more complex list, namely
\begin{lstlisting}
gap> r3:=AddNP(AddNP(MulNP(x,y),MulNP(y,x),1,1),u,1,-b);
[[ [ 2, 1 ], [ 1, 2 ], [  ] ], [ 1, 1, -b ] ]
\end{lstlisting}
This one is made out of two lists: a list of coefficients \lstinline{[ 1, 1, -b ]} that multiply the elements in the list of variables \lstinline{[ [ 2, 1 ], [ 1, 2 ], [  ] ]}. It is important to notice that this second list of variables contains three (as predicted by the number of coefficients) elements, each of which is, once again, a list: \lstinline{ [ 2, 1 ]},\lstinline{ [ 1, 2 ]} and  \lstinline{  [  ]}. The empty list is the unit in the algebra (as we can see in the definition  \lstinline{u:=[[[]],[1]]} above) and we see that if a list contains more than one variable, then this indicates the multiplication, or concatenation, of the variables inside it.

Finally, we see that the relations defining the algebra is a list  \texttt{K:=[r1,r2,r3]} with the three relations.

Thus, we can retrive a given relation $rn$ with \lstinline{K[n]}, the coefficients on it with \lstinline{K[n][2]}, the variables with \lstinline{K[n][1]}, or the (combination of) variables at position $m$ with \lstinline{K[n][1][m]}, e.g.:
\begin{lstlisting}
gap> K[2][1][1];
[ 2, 2 ]
\end{lstlisting}
This becomes particularly handy when dealing with long lists of complicated relations, as the ones that can appear once we compute the Gr\"obner basis associated to an ideal.
\end{example}

\section{The NP setting}

From the example above, we learn a couple of functions, e.g.~\lstinline{GBNP.ConfigPrint} that gives the name in which the variables will be printed on the screen. We see that the $n$th variable is called with \texttt{ [[[n]],[1]]}, while the unit of the algebra is \lstinline{ [[[ ]],[1]]}. We can multiply the $n$th and $m$th variables via \lstinline{ [[[n,m]],[1]]} or with the two-input function \lstinline{MulNP}. Similarly, we can form linear combinations of the variables $n$th and $m$th with \lstinline{ [[[n],[m]],[a,b]]}, where $a,b$ stand for a pair of coefficients on the base field. Similarly for products (concatenations of variables).
This can also be achieved with the  four-input function \lstinline{AddNP} (or a concatenation of this same function). We can ask for the standard algebraic input of a combination with the command \lstinline{PrintNP}:
\begin{lstlisting}
gap> GBNP.ConfigPrint("x","y");;
gap> PrintNP([[[1]],[1]]);
 x 
\end{lstlisting}
Once we have computed the Gr\"obner basis associated to a given ideal, we can ask for a reduced expression on the corresponding quotient with \lstinline{StrongNormalFormNP}. In the context of Example \ref{example:Clifford_revisited}:
\begin{lstlisting}
gap> G:=SGrobner(K);;
gap> StrongNormalFormNP(r1,G);
[ [  ], [  ] ]
gap> xyx:=StrongNormalFormNP(MulNP(MulNP(x,y),x),G);
[ [ [ 2 ], [ 1 ] ], [ -a, b ] ]
gap> PrintNP(xyx);
 -ay + bx 
\end{lstlisting}
It is also possible to ask for a list of relations to be displayed in algebraic setup:
\begin{lstlisting}
gap> PrintNPList(G);                                
 x^2 + -a 
 yx + xy + -b 
 y^2 + -a 
\end{lstlisting}

\subsection{The truncated Gr\"obner basis}
When dealing with algebras defined by homogeneous relations with respect to a certain assignment of weights to the generators, it is also possible to compute a truncated Gr\"obner basis. In some cases where the full Gr\"obner basis cannot be computed or takes too much time, this truncated variant can be also used to make computations. It is called with  \lstinline{SGrobnerTrunc(K,p,n)} where $K$ is a list of relations, $p$ is a weight vector and $n$ is     
% FIXME
\begin{lstlisting}
gap> A := FreeAssociativeAlgebraWithOne(Rationals, "a", "b", "c");;
gap> a := A.a;;
gap> b := A.b;;
gap> c := A.c;;
gap> one := One(A);;
gap> rels := [ a^4,  b^5, b*a-c^3 ];;
gap> K := GP2NPList(rels);;
gap> G := SGrobner(K);;
gap> Gs:= SGrobnerTrunc(K,4,[1,2,1]);;
gap> r := a^4;;
gap> PrintNP(StrongNormalFormNP(\
> [ [ [ 1, 1, 1, 1 ] ], [ 1 ] ],Gs));
0
\end{lstlisting}

\section{Tensor products}

A common question is how to deal with tensor products, as in the following

\begin{question}\label{question-taft} 
	Let us fix a positive integer $N$ and an $N$th root of 1, $q$.  Let $H$ be the algebra
	generated by elements $x$ and $g$, with relations $gx=q\,xg$, $g^N=1$ and
	$x^N=0$. Define $\Delta(g)=g\otimes g$, $\Delta(x)=x\otimes 1+g\otimes x$.
	Show that $\Delta$ defines an algebra map $H\to H\otimes H$.
\end{question}

A possible answer is to ``double'' the algebra, as the tensor product $H\otimes H$ is the algebra generated by $x,y$ and $g,h$, with relations
\begin{align*}
x^N&=y^N=0, & g^N&=h^N=1, & gx&=q\,xg, & hy&=q\,yh, \\
hx&=xh, & hg&=gh, & yx&=xy, & yg&=gy.
\end{align*}
Here $x$ represents the term $x\otimes 1$, while $y$ is $1\otimes x$; same for $g=g\otimes 1,h=1\otimes g$. Then we compute the Gr\"obner basis \lstinline{G} of this ideal and we can define $\Delta(x)$ as
\begin{lstlisting}
gap> Dx:=AddNP(x,MulNP(g,y),1,1);;
\end{lstlisting}
We can answer the question affirmatively by checking that $\Delta(x)^N=0$:
\begin{lstlisting}
gap> MulNP(Dx,MulNP(Dx,MulNP(..., Dx),...));
\end{lstlisting}
Alternatively, we may consider just the ideal generated by relations
\begin{align*}
gx&=q\,xg, & hy&=q\,yh, & hx&=xh, & hg&=gh, & yx&=xy, & yg&=gy.
\end{align*}
and then checking that $\Delta(x)^N=x^N\otimes 1+g^N\otimes y^N$.

\medskip

The same strategy works for twisted tensor products:

\begin{question}\label{question-taft-braided}. Let $N,q$ be as in Question \ref{question-taft}. 
	Let $T$ be the algebra generated by an element $x$ such that $x^N=0$; then $T\hat{\otimes} T$ is an algebra with 
	\[
	(x^a\otimes x^b)(x^c\otimes x^d)=q^{bc}\,x^{a+c}\otimes x^{b+d}.
	\] Define $\Delta(x)=x\otimes 1+1\otimes x$. Show that $\Delta$ defines an algebra map $T\to T\hat{\otimes} T$.
\end{question}
Again, we ``double'' the algebra, as the tensor product $T\hat{\otimes} T$ is the algebra generated by $x,y$ with relations
\begin{align*}
x^N&=y^N=0, & yx&=q\,xy.
\end{align*}
Here $x=x\hat{\otimes} 1$, $y=1\hat{\otimes}x$. We define $\Delta(x)$ as
\begin{lstlisting}
gap> Dx:=AddNP(x,y,1,1);;
\end{lstlisting}
and proceed as in the previous example.

\medskip

As well, a similar idea can be used to deal with tensor products of different algebras.
\begin{question}
 Let $q$, $T$ be as in Question \ref{question-taft-braided}. Let $S$ be the algebra generated by an element $y$ such that $y^N=1$; then $S\hat{\otimes} T$ is an algebra with 
	\[
	(y^a\otimes x^b)(y^c\otimes x^d)=q^{bc}\,y^{a+c}\otimes x^{b+d}.
	\] 
	Show that $\rho(y)=y\otimes 1+1\otimes x$ defines a braided right $T$-comodule algebra structure $S\to S\hat{\otimes} T$ on $S$. 
\end{question}
In this case, $S\hat{\otimes} T$ is the algebra generated by $x,y$ with relations
\begin{align*}
x^N&=0, & y^N&=1, & xy&=q\,yx.
\end{align*}
Here $y=y\hat{\otimes} 1$, $x=1\hat{\otimes}x$. Then $\rho(y)$ is
\begin{lstlisting}
gap> ry:=AddNP(y,x,1,1);;
\end{lstlisting}
Again, we proceed as above.





\section{Problems}

\begin{prob}
Write a function to compute the list of relations of $\mathcal{E}_n$.
\end{prob}

\begin{prob}
	Compute a linear basis of the center of the algebra $\mathcal{E}_3$ of
	Example~\ref{example:FK3}. 
\end{prob}

\begin{prob}
	Is $\mathcal{E}_3$ semisimple?
\end{prob}

\begin{prob}
	\label{problem:FK4}
	Prove that the algebra $\mathcal{E}_4$ is finite-dimensional.
\end{prob}

\begin{prob}
	\label{problem:FK5}
	Prove that the algebra $\mathcal{E}_5$ is finite-dimensional.
\end{prob}

\begin{prob}
	\label{problem:FK6}
	Prove that the Hilbert series of $\mathcal{E}_6$ is given by
	\[
	H(t)=1+15t+125t^2+765t^3+3831t^4+16605t^5+64432t^6+228855t^7+\cdots
	\]
\end{prob}

\begin{prob}\label{prob:PBW-def}
Find all the PBW deformations of $\mathcal{E}_3$; i.e.~find necessary and sufficient conditions on scalars $\alpha_1,\alpha_2,\alpha_3$ and $\beta_1,\beta_2$ so that the algebra $\widetilde{\mathcal{E}}_3\coloneqq \mathcal{E}_3(\alpha_1,\alpha_2,\alpha_3,\beta_1,\beta_2)$ generated by $a,b,c$ and relations
\begin{align*}
a^2=\alpha_1,&& 
b^2=\alpha_2,&&
c^2=\alpha_3,&&
ab+ca+bc=\beta_1,&&
ba+ac+cb=\beta_2.
\end{align*}
is such that the graded algebra $\gr \widetilde{\mathcal{E}}_3$  associated  to the natural filtration $(|a|=|b|=|c|=1)$ satisfies  $\gr \widetilde{\mathcal{E}}_3\simeq \mathcal{E}_3$.
\end{prob}



\begin{prob}
	Show that the algebras $\widetilde{\mathcal{E}}_3$ in Problem \ref{prob:PBW-def} are twisted right comodule algebras over $\mathcal{E}_3$.
\end{prob}



