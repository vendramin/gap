\chapter{The Yang--Baxter equation}

En 1990, Drinfeld \cite{MR1183474} propuso estudiar la versión conjuntista de
la ecuación de Yang--Baxter. El problema es el siguiente: dado un conjunto
finito $X$, se quieren encontrar biyecciones $c\colon X\times X\to X\times X$
tales que
\[
    (c\times\id)(\id\times c)(c\times\id)=(\id\times c)(c\times\id)(\id\times c).
\]

En \cite{MR1722951}, Etingof, Schedler y Soloviev estudiaron el problema en el
caso en que $c$ es una involución (es decir $c^2=\id$). En ese trabajo y con
ayuda de la computadora se construyeron todas las soluciones involutivas sobre
conjuntos de cardinal $\leq 8$. En 2005, Rump \cite{MR2132760} introdujo una
estructura no asociativa que permite describir soluciones involutivas de la
ecuación de Yang--Baxter: \emph{cycle sets}. En esta sección daremos las
nociones básicas y escribiremos funciones que nos permitan trabajar con estas
estructuras.

\subsection{Nociones básicas}

\begin{block}
	
\end{block}

\subsection{Cycle sets}

\begin{block}
Un \textbf{cycle set} es un conjunto $X$ junto con una operación
$\cdot\colon X^2\to X$ que verifica $(x\cdot y)\cdot (x\cdot z)=(y\cdot
x)\cdot (y\cdot z)$ para todo $x,y,z\in X$. La siguiente función 
devuelve la sucesión de permutaciones que define un cycle set:
\begin{lstlisting}
gap> CS_Permutations := function(cycleset)
> return List([1..Size(cycleset.matrix)], \
> i->PermList(cycleset.matrix[i]));
> end;
function( cycleset ) ... end
\end{lstlisting}
\end{block}

\begin{block}
	Si $S\colon X^2\to X^2$ es una solución involutiva, escribimos
	$S(x,y)=(f(x,y),y*x)$ para alguna $f\colon X\times X\to X$.  Como $S$ es
	inversible, para cada $y\in X$ la función $x\mapsto y*x$ es inversible. Se
	define entonces la operación $\cdot\colon X^2\to X$ por $x\cdot y=z$ si y
	sólo si $x*z=y$, y luego $(X,\cdot)$ es un cycle set. En el código siguiente
	escribimos una función que convierte soluciones involutivas de la ecuación
	de trenzas en cycle sets:

	\begin{lstlisting}
solution2cycleset := function(solution)
  local p,m;
  m := [];
  for p in solution.r_actions do
    Add(m, ListPerm(Inverse(PermList(p)),solution.size));
  od;
  return rec(size := solution.size, matrix := m);
end;
	\end{lstlisting}

	Recíprocamente, si $(X,\cdot)$ es un cycle set, entonces 
	\[
		S(x,y)=\left( (y*x)\cdot y,y*x \right),
	\]
	donde $x\cdot z=y$ si y sólo si $x*y=z$, es una solución involutiva. 
\end{block}
