
%

\begin{example}
Let $A=\{2,\dots,100\}$ and let $\sim$ be the binary
relation given by $n\sim m$ if and only if $\gcd(n,m)=1$.
We construct the binary relation
and then check whether the relation is
reflexive, (anti)symmetric and transitive.

\begin{lstlisting}
gap> A := [2..100];;
gap> l := [];;
gap> for x in A do
> for y in A do
> if Gcd(x,y)>1 then
> Add(l, Tuple([x,y]));
> fi;
> od;
> od;
gap> rel := BinaryRelationByElements(Domain(A), l);;
gap> IsReflexiveBinaryRelation(rel);
true
gap> IsSymmetricBinaryRelation(rel);
true
gap> IsAntisymmetricBinaryRelation(rel);
false
gap> 1IsTransitiveBinaryRelation(rel);
false
\end{lstlisting}

To finish, we want to compute the number
of elements that are related to $12$.

\begin{lstlisting}
gap> Number(A, x->Gcd(x,12)>1);
67
\end{lstlisting}

Can you compute the number of elements
related to 12 in the case
where $A=\{2,\dots,100000\}$?
\end{example}

\begin{example}
For $g$ a polynomial such that $g(0)\ne 0$
consider the sequence of polynomials
given by $f_1=Xg$ and
$f_{n+1}=(Xf'_n)^n$. Compute the multiplicity
of $0$ as a root of $f_n$.

Let us fix $g=1$ and construct the sequence up to
some bound, say only five terms.
\begin{lstlisting}
gap> N := 5;;
gap> x := Indeterminate(Rationals, "x");;
gap> g := 1;;
gap> f := [x*g];;
gap> for n in [2..N] do
> Add(f, (x*Derivative(f[n-1], x))^(n-1));
> od;
gap> f;
[ x, x, x^2, 8*x^6, 5308416*x^24 ]
\end{lstlisting}

One can make experiments and use different values of $g$. It is an exercise to
the reader to prove that the multiplicity of zero in $f_n$ is $n!$,
independently of $g$.
\end{example}

\section{Promislow's group}


\begin{definition}
A group $G$ has the \emph{unique product property} if 
for all finite non-empty subsets $A$ and $B$ of $G$ 
there exists $x\in G$ that can be written uniquely as
$x = ab$ with $a\in A$ and $b\in B$.
\end{definition}

Groups with the unique product property satisfy 
Kaplansky's unit conjecture. Moreover, 
left-ordered groups satisfy 
the unique product property. However, the converse
does not hold. Promislow provided the first counterexample.

\begin{example}
    The group $G=\langle a,b:a^{-1}b^2a=b^{-2},b^{-1}a^2b=a^{-2}\rangle$
    does not have the unique product property. 
    It is known that $G$ is torsion-free, see... 
    Let 
    \begin{multline}
    \label{eq:Promislow}
    S=\{ a^2b,
    b^2a,
    aba^{-1},
    (b^2a)^{-1},
    (ab)^{-2},
    b,
    (ab)^2x,
    (ab)^2,
    (aba)^{-1},\\
    bab,
    b^{-1},
    a,
    aba,
    a^{-1}
    \}.
    \end{multline}
    We use \textsf{GAP} and the representation $G\to\GL(4,\Q)$ given by 
    \[
a\mapsto\begin{pmatrix}
1 & 0 & 0 & 1/2\\
0 & -1 & 0 & 1/2\\
0 & 0 & -1 & 0\\
0 & 0 & 0 & 1
\end{pmatrix},
\quad
b\mapsto\begin{pmatrix}
-1 & 0 & 0 & 0\\
0 & 1 & 0 & 1/2\\
0 & 0 & -1 & 1/2\\
0 & 0 & 0 & 1
\end{pmatrix}
\]
    to check that 
    $G$ does not have
    unique product property, as each 
    \[
    s\in S^2=\{s_1s_2:s_1,s_2\in S\}
    \]
    admits at least two different decompositions of the 
    form $s=xy=uv$ for $x,y,u,v\in S$. 
    We first create the matrix representations of $a$ and $b$.
\begin{lstlisting}
gap> a := [[1,0,0,1/2],[0,-1,0,1/2],[0,0,-1,0],[0,0,0,1]];;
gap> b := [[-1,0,0,0],[0,1,0,1/2],[0,0,-1,1/2],[0,0,0,1]];;
\end{lstlisting}
    Now we create
    a function that produces the set $S$.
\begin{lstlisting}
gap> Promislow := function(x, y)
> return Set([
> x^2*y,
> y^2*x,
  x*y*Inverse(x),
  (y^2*x)^(-1),
  (x*y)^(-2),
  y,
  (x*y)^2*x,
  (x*y)^2,
  (x*y*x)^(-1),
  y*x*y,
  y^(-1),
  x,
  x*y*x,
  x^(-1)
]);
end;;
\end{lstlisting}
So the set $S$ of \eqref{eq:Promislow} 
will be \lstinline{Promislow(a,b)}. We now
create a function that checks whether
every element of a Promislow subset 
admits more than one representation.
\begin{lstlisting}
gap> is_UPP := function(S)
> local l,x,y;
> l := [];
> for x in S do
> for y in S do
> Add(l,x*y);
> od;
> od;
> if ForAll(Collected(l), x->x[2] <> 1) then
> return false;
> else
> return fail;
> fi;
> end;;
\end{lstlisting}
Finally, we check whether every element of 
$S$ admits more than one representation.
\begin{lstlisting}
gap> S := Promislow(a,b);;
gap> is_UPP(S);
false
\end{lstlisting}
This completes the proof. 
\end{example}
