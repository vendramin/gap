\chapter{Nichols algebras}

\section{Braidings and Nichols algebras}

We start with the definition of Nichols algebras. For that purpose we need
first to define braided vector spaces. 

\begin{definition}
	A \emph{braided vector space} is a pair $(V,c)$, where $V$ is a vector
	space and $c\in\GL(V\otimes V)$ is a
	solution of the \emph{braid equation}:
	\[
	(c\otimes\id)(\id\otimes\, c)(c\otimes\id)
	=(\id\otimes\, c)(c\otimes\id)(\id\otimes\, c).
	\]
\end{definition}

%\begin{example}[braided vector spaces of diagonal type]
%	Let $V$ be a vector space with basis $x_{1},x_{2},...,x_{n}$. For $i,j\in\{1,\dots,n\}$ let
%	$q_{ij}\in\C^\times$ and 
%	let 
%	\[
%		c(x_{i}\otimes x_{j})=q_{ij}x_{j}\otimes x_{i}.  
%	\]
%	Then
%	$(V,c)$ is a braided vector space.
%\end{example}
%
%\begin{example}
%	Let $G$ be a finite group and $V=\C G$ be the complex vector space with basis
%	$\{g:g\in G\}$. Let \[
%		c(g\otimes h)=ghg^{-1}\otimes g
%	\]
%	for $g,h\in G$.  Then $(V,c)$ is a braided vector space. 
%\end{example}

\begin{example}
\label{exa:Yamane}
Let $V$ be a vector space with basis $\{a,b\}$. Let $q$ be a variable and let $c\in\GL(V\otimes V)$ be given by
\begin{align*}
	c(a\otimes a)=q\,b\otimes a,&&
	c(a\otimes b)=q\,a\otimes a,&&
	c(b\otimes a)=q\,b\otimes b,&&
	c(b\otimes b)=q\,a\otimes b.
\end{align*}
Let us check that $c$ is a braiding:
\begin{lstlisting}
gap> field := FunctionField(Rationals, 1);;
gap> q := IndeterminatesOfFunctionField(field)[1];;
gap> c := [[0,0,q,0],[q,0,0,0],[0,0,0,q],[0,q,0,0]]*One(field);;
gap> id := IdentityMat(2)*One(field);;
gap> c1 := KroneckerProduct(c, id);;
gap> c2 := KroneckerProduct(id, c);;
gap> c1*c2*c1=c2*c1*c2;
true
\end{lstlisting}
\end{example}

\begin{example}
	Let $V$ be a vector space with basis $x,y$. For $q\in\C^{\times}$ let $c$ be such that 
	\begin{align*}
		&c(x\otimes x)=x\otimes x, && 
		c(y\otimes y)=y\otimes y, \\
		&c(x\otimes y)=q\, y\otimes x, && 
		c(y\otimes x)=q\, x\otimes y+(1-q^2)y\otimes x.
	\end{align*}
	Then $(V,c)$ is a braided vector space. Let us check that $c$ is invertible
	and satisfies the braid equation:
\begin{lstlisting}
gap> field := FunctionField(Rationals, 1);;
gap> q := IndeterminatesOfFunctionField(field)[1];;
gap> c := [\
> [1,0,0,0],\
> [0,0,q,0],\
> [0,q,1-q^2,0],\
> [0,0,0,1]]*One(field);;
gap> Display(Inverse(c));
[ [       1,       0,       0,       0 ],
  [       0,  1-q^-2,    q^-1,       0 ],
  [       0,    q^-1,       0,       0 ],
  [       0,       0,       0,       1 ] ]
gap> id := IdentityMat(2)*One(field);;
gap> c1 := KroneckerProduct(c, id);;
gap> c2 := KroneckerProduct(id, c);;
gap> c1*c2*c1=c2*c1*c2;
true
\end{lstlisting}
\end{example}

%We will work with a particular family of braided vector spaces.
%
%\begin{definition}
%	Let $G$ be a group. 
%	A \textbf{Yetter-Drinfeld module} over $G$ is a $\C G$-module
%	$V=\oplus_{g\in G}V_{g}$ such that $hV_{g}\subseteq V_{hgh^{-1}}$ for all
%	$g,h\in G$. 
%\end{definition}
%
%\begin{exercise}
%	Let $G$ be a group and $V$ be a Yetter-Drinfeld module over $G$. Prove that
%	the map $c:V\otimes V\to V\otimes V$ given by  $c(u\otimes v)=xv\otimes u$
%	whenever $\deg u=x$, is a solution of the braid equation.
%\end{exercise}
%
%Over the complex numbers, the category of Yetter-Drinfeld modules over $G$
%is semisimple.  Furthermore, simple Yetter-Drinfeld modules over $G$ are
%parametrized by pairs $(g^G,\rho)$, where $g^G$ denotes the conjugacy class of $g\in G$  
%and $(\rho,M)$ is an irreducible representation of the centralizer
%$C_G(g)$. 
%
%Let us describe the simple Yetter-Drinfeld modules over $G$. Let $\{x_1,\dots,x_n\}$
%be a set of representatives of left cosets of $C_G(g)$ in $G$.  Then the simple
%Yetter-Drinfeld modules over $G$ are
%\[
%M(g^G,\rho)=\mathrm{Ind}_{C_G(g)}^G\rho=\bigoplus_{i=1}^n x_i\otimes_{\mathbb{C}C_G(g)}M
%\]
%with the induced action $y(x\otimes m)=yx\otimes m$ for $x,y\in G$ and $m\in
%M$, and the coaction $\delta(x_i\otimes m)=x_igx_i^{-1}\otimes(x_i\otimes
%m)$ for $m\in W$. The braiding is
%\[
%c\left((x_i\otimes m)\otimes(x_j\otimes m') \right)=(x_igx_i^{-1}x_j\otimes m')\otimes(x_i\otimes m).
%\]
%Thus every simple Yetter-Drinfeld module over $G$ can be written as
%$V=\oplus_{x\in g^G}V_x$, where $V_x=\{v\in V:\delta(v)=x\otimes v\}$ and
%$V_g=1\otimes M$. For all $x\in g^G$, $V_x$ is a simple $C_G(x)$-module and
%$yV_x\subseteq V_{yxy^{-1}}$ for all $y\in G$.
%
%%\section{Racks}
%%
%%Let us study the combinatorics behind a Yetter-Drinfeld module. 
%%
%%\begin{definition}
%%	A \emph{rack} is a pair $(X,\triangleright)$, where $X$ is a set and
%%	$X\times X\to X$, $(x,y)\mapsto x\triangleright y$, is an operation 
%%	such that the maps $\varphi_x\colon
%%	X\times X\to X$, given by $y\mapsto x\triangleright y$, are bijective for
%%	each $x\in X$, and $x\triangleright
%%	(y\triangleright z)=(x\triangleright y)\triangleright (x\triangleright z)$
%%	for all $x,y,z\in X$.
%%\end{definition}
%%
%%\begin{exercise}
%%	Let $X$ be a set with an operation $X\times X\to X$, $(x,y)\mapsto x\triangleright y$. 
%%	Prove that the map $r\colon X\times X\to X\times X$, $r(x,y)=(x\triangleright y,x)$,
%%	is a solution of the set-theoretical braid equation
%%	\[
%%	(r\times\id)(\id\times r)(r\times\id)
%%	=(\id\times r)(r\times\id)(\id\times r)
%%	\]
%%	if and only if $(X,\triangleright)$ is a rack.
%%\end{exercise}
%%
%%Racks produce braided vector spaces.  Let $X$ be a rack, $V=\C X$
%%be the vector space with basis $\{x:x\in X\}$ and $c(x\otimes
%%y)=x\triangleright y\otimes x$.  Then $(V,c)$ is a braided vector space. 
%%
%%
%%\begin{example}
%%	Let $G$ be a group and $X$ be a union of conjugacy classes. Then $X$ with
%%	$x\triangleright y=xyx^{-1}$ for all $x,y\in X$ is a quandle. 
%%\end{example}
%%
%%\begin{example}
%%	Let $X=\Z/n$ and $x\triangleright y=2x-y$. Then $(X,\triangleright)$ is a rack.
%%\end{example}
%%
%%\begin{example}
%%	Let $X$ be an abelian group, $g\in\Aut(X)$ and $x\triangleright y=(\id-g)(x)+g(y)$. 
%%	Then $(X,\triangleright)$ is a rack.
%%\end{example}
%%
%%\begin{exercise}
%%	\label{xca:2cocycle}
%%	Let $X$ be a rack and $q\colon X\times X\to\C^{\times}$ be a map. Let $V=\C
%%	X$ and 
%%	\[
%%		c(x\otimes y)=q(x,y)x\triangleright y\otimes x.
%%	\]
%%	Prove that $(V,c)$ is a braided vector space if and only if 
%%	\begin{equation}
%%		\label{eq:2cocycle}
%%		q(x,y\triangleright z)q(y,z)=q(x\triangleright y,x\triangleright z)q(x,z)
%%	\end{equation}
%%	for all $x,y,z\in X$. 
%%\end{exercise}
%%
%%
%%\begin{example}	
%%	Let $-X=(123)^{\Alt_4}$ be the quandle associated with the conjugacy class
%%	of $(123)$ in the alternating group $\Alt_4$. The map $q\colon X\times
%%	X\to\C$ given by
%%	\begin{equation}
%%		\label{eq:2cocycle}
%%		\begin{array}{c|cccc}
%%			& (243) & (123) & (134) & (142)\\
%%			\hline (243) & \omega & \omega & \omega & \omega\\
%%			(123) & \omega & \omega & -\omega & -\omega\\
%%			(134) & \omega & -\omega & \omega & -\omega\\
%%			(142) & \omega & -\omega & -\omega & \omega
%%		\end{array}
%%	\end{equation}
%%	where $\omega\in\C$ is a primitive $n$-th root of $1$, is a $2$-cocycle of
%%	$X$. 
%%\end{example}
%%
%%\begin{example}
%%	\label{exa:FK}
%%	Let $n\geq3$ and $X_{n}=(12)^{\Sym_{n}}$.  The map 
%%	$\chi\colon X\times X\to\C$ given by
%%	\[
%%		\chi(\sigma,\tau)=\begin{cases}
%%		1 & \text{if }\sigma(i)<\sigma(j),\\
%%		-1 & \text{otherwise,}
%%	\end{cases}
%%	\]
%%	where $\tau=(ij)$, $i<j$, is a $2$-cocycle of $X_{n}$.
%%\end{example}
%%
%%The braided vector spaces of Exercise~\ref{xca:2cocycle} are called of type $(X,q)$. 
%%Braided vector spaces coming from 
%%Yetter-Drinfeld modules over groups are braided vector spaces of type $(X,q)$,
%%for some rack $X$ and some $q$ such that~\eqref{eq:2cocycle} holds.
%
Recall that the braid group $\B_n$ is the group with generators
$\sigma_{1},...,\sigma_{n-1}$ and relations
\begin{align*}
	&\sigma_{i}\sigma_{i+1}\sigma_{i}=\sigma_{i+1}\sigma_{i}\sigma_{i+1} && \text{for }1\leq i\leq n-2,\\
	&\sigma_{i}\sigma_{j}=\sigma_{j}\sigma_{i} && \text{for \ensuremath{|i-j|>1}.}
\end{align*}
Braided vector spaces produce representations of the braid group  $\B_n$ in the following way. 
Let $(V,c)$ be a braided vector space, and $n\in\N$. Define
\[
c_{i}=\id_{V^{\otimes(i-1)}}\otimes\, c\otimes\id_{V^{\otimes(n-i-1)}}\in\Aut(V^{\otimes n})
\]
for $1\leq i\leq n-1$, i.e.,
\[
c_{i}\cdot(v_{1}\otimes\cdots\otimes v_{n})=v_{1}\otimes\cdots\otimes v_{i-1}\otimes\, c(v_{i}\otimes v_{i+1})\otimes v_{i+2}\otimes\cdots\otimes v_{n}.
\]
The operators $c_{i}$, $1\leq i\leq n-1$, satisfy the defining-relations of the
braid group and hence 
\[
	\rho_{n}:\B_{n}\to\Aut(V^{\otimes n}),
	\quad
	\sigma_i\mapsto c_i,
\]
is a representation of $\B_{n}$ on $V^{\otimes n}$.

The symmetric group $\Sym_{n}$ can be presented with generators 
$\tau_{1},...,\tau_{n-1}$ and relations
\begin{align*}
	&\tau_{i}\tau_{i+1}\tau_{i}=\tau_{i+1}\tau_{i}\tau_{i+1} &  & 1\leq i\leq n-2,\\
	&\tau_{i}\tau_{j}=\tau_{j}\tau_{i} &  & \ensuremath{|i-j|>1},\\
	&\tau_{i}^{2}=1 &  & 1\leq i\leq n-1.
\end{align*}
Thus there exists a surjection $\B_{n}\to\Sym_{n}$ defined by
$\sigma_{i}\mapsto\tau_{i}$. 

\begin{lemma}
	There exists a set-theoretical section
	$\mu:\mathbb{S}_{n}\to\mathbb{B}_{n}$, $\tau_{i}\mapsto\sigma_{i}$, such
	that 
	\[
	\mu(xy)=\mu(x)\mu(y) \quad \text{if}\quad 
 \mathrm{length}(xy)=\mathrm{length}(x)+\mathrm{length}(y). 
	\]
\end{lemma}

\begin{proof}
	See \cite{bibid}.
\end{proof}

Let $(V,c)$ be a braided vector space. The \emph{Nichols algebra} of $(V,c)$ is 
\[
\NA(V,c)=\C\oplus V\oplus \bigoplus_{n\geq2}V^{\otimes n}/\ker\mathfrak{S}_n,
\]
where 
\[
\mathfrak{S}_{n}=\sum_{\sigma\in\Sym_{n}}\rho_{n}\mu(\sigma)\in \End V^{\otimes n}, \qquad n\geq 2,
\]
is the \emph{quantum symmetrizer} of degree $n$.

\begin{example}
	Let us compute some quantum symmetrizers:
	\begin{align*}
		&\mathfrak{S}_2 = \id+c,\\
		&\mathfrak{S}_3 = \id+c_{1}+c_{2}+c_{1}c_{2}+c_{2}c_{1}+c_{1}c_{2}c_{1},
	\end{align*}
	where $c_1=c\otimes\id$ and $c_2=\id\otimes\, c$. 
\end{example}

Now two basic examples of Nichols algebras:

\begin{example}
	Let $V$ be a vector space and let $\tau:V\otimes V\to V\otimes V$
	be the linear map defined by $x\otimes y\mapsto y\otimes x$. The Nichols
	algebra of the braided vector space $(V,\tau)$ is the Symmetric Algebra
	$S(V)$. The Nichols algebra of the braided vector space $(V,-\tau)$ is the
	Exterior Algebra $\Lambda(V)$.
\end{example}

\section{Examples and computations}

We need to write some code to compute kernel symmetrizers and compute their
dimensions.  Let us first start with the braiding. Let $(V,c)$ be a braided
vector space with a fixed basis. For any given $i\in\{1,\dots,n-1\}$ the
following function returns the matrix of the map $c_i\colon V^{\otimes n}\to
V^{\otimes n}$:

\begin{lstlisting}
gap> Braiding := function(c, i, n)
> local d;
> d := Sqrt(Size(c));
> return KroneckerProduct(\
> KroneckerProduct(IdentityMat(d^(i-1),d^(i-1)),c),\
> IdentityMat(d^(n-i-1),d^(n-i-1)));
> end;
function( c, i, n ) ... end
\end{lstlisting}

Now we need a function that computes the map $c_ic_{i-1}\cdots c_1\colon V^{\otimes
n}\to V^{\otimes n}$: 

\begin{lstlisting}
gap> SeveralBraidings := function(c, i, n)
> local m, k, d;
> d := Sqrt(Size(c));
> m := Braiding(c, 1, n);
> for k in [2..i] do
> m := Braiding(c, k, n)*m;
> od;
> return m;
> end;
function( c, i, n ) ... end
\end{lstlisting}

The following function computes the quantum symmetrizer recursively:
\begin{lstlisting}
gap> Symmetrizer := function(c, n)
> local i, m, d;
> d := Sqrt(Size(c));
> if n=1 then
> return IdentityMat(d,d);
> fi;
> m := IdentityMat(d^n, d^n);
> for i in [1..n-1] do
> m := m+SeveralBraidings(c, i, n);
> od;
> return m*KroneckerProduct(IdentityMat(d,d), \
> Symmetrizer(c, n-1));
> end;
function( c, n ) ... end
\end{lstlisting}

%To improve our recursive function we use the package
%\lstinline{ToolsForHomalg}, written by Barakat, Gutsche and Lange-Hegermann. We
%first need to load the package:
%
%\begin{lstlisting}
%gap> LoadPackage( "ToolsForHomalg" );
%\end{lstlisting}
%
%Now we rewrite our function \lstinline{Symmetrizer} trying to avoid recomputing
%already known symmetrizers. We thank Sebastian Posur for this trick.
%
%\begin{lstlisting}
%gap> Symmetrizer := FunctionWithCache( function(c, n)
%> local i, m, d;
%> d := Sqrt(Size(c));
%> if n=1 then
%> return IdentityMat(d,d);
%> fi;
%> m := IdentityMat(d^n, d^n);
%> for i in [1..n-1] do
%> m := m+SeveralBraidings(c, i, n);
%> od;
%> return m*KroneckerProduct(IdentityMat(d,d), \
%> Symmetrizer(c, n-1));
%> end : Cache := "crisp");
%function( arg... ) ... end
%\end{lstlisting}

The functions we wrote can now be used to compute some dimensions of Nichols
algebras. Here is the code:
\begin{lstlisting}
gap> ComputeDimension := function(c, n)
> return RankMatDestructive(Symmetrizer(c, n));
> end;
function( c, n ) ... end
\end{lstlisting}

Once we have the quantum symmetrizers it is also possible to compute defining
relations. The following code returns relations in \lstinline{gbnp} format:

\begin{lstlisting}
gap> ComputeRelations := function(c, n)
> local m, ns, i, j, a, b, res;
> m := Symmetrizer(c, n);
> TransposedMatDestructive(m);
> ns := NullspaceMatDestructive(m);
> res := [];
> for i in [1..Size(ns)] do
> a := [];
> b := [];
> for j in [1..Size(ns[1])] do
> if ns[i][j] <> 0 then
> Add(a, EnumeratorOfTuples([1..Sqrt(Size(c))], n)[j]);
> Add(b, ns[i][j]);
> fi;
> od;
> Add(res, [a,b]);
> od;
> return res;
> end;
function( c, n ) ... end
\end{lstlisting}

Let us explore some examples of Nichols algebras. 

\begin{example}[Yamane]
Let $(V,c)$ be the braided vector space of Example~\ref{exa:Yamane} with $q$ a
primitive third root of one. 
Then $\dim\NA(V,c)=108$. The Hilbert series is
\begin{multline*}
H(t)=1+2t+ 4t^2+ 6t^3+ 9t^4+ 12t^5+ 13t^6
+ 14t^7\\+ 13t^8+ 12t^9+ 9t^{10}+ 6t^{11}+ 4t^{12}+ 2t^{13}+t^{14}.
\end{multline*}
Here is the code:
\begin{lstlisting}
gap> c := [[0,E(3),0,0],[0,0,0,E(3)],[E(3),0,0,0],[0,0,E(3),0]];;

\end{lstlisting}
There are two relations of degree three and two of degree six:
\begin{align*}
&a^2b -q^2ab^2 -q^2ba^2 + b^2a=
a^3 + aba -qab^2 -qba^2 + bab + b^3=0,\\
&a^5b + a^4ba + a^3ba^2 + a^2ba^3 + aba^4 + ba^5=
a^6=0.
\end{align*}
The code for computing the relations in degree three and show them in a human
readable form is the following:
\begin{lstlisting}
gap> LoadPackage("gbnp");;
gap> PrintNPList(ComputeRelations(c,3));
 a^2b + -E(3)^2ab^2 + -E(3)^2ba^2 + b^2a 
 a^3 + aba + -E(3)ab^2 + -E(3)ba^2 + bab + b^3 
\end{lstlisting}
This braided vector space is of diagonal type with respect to the basis
$\{a+b,a-b\}$. %The braiding matrix has entries q, q, -q, -q and Dynkin diagram
% q -q^2 -q. 
%The Nichols algebra is a special case of the one parameter
%family with Dynkin diagram q r^{-1} r, where $q$ is a primitive third root
%of 1. 
This family was first found by Yamane as his $\Z_3$-graded quantum group,
see~\cite{MR2317113}.
\end{example}

\begin{example}[Rowell]
Let $V$ be a vector space with basis $\{a,b\}$. Let
$\lambda=\frac{-1+i}{2}$ and let $c$ be the linear map given by
\begin{align*}
c(a\otimes a)=\lambda a\otimes a-\lambda b\otimes b, && 
c(a\otimes b)=\lambda a\otimes b-\lambda b\otimes a,\\
c(b\otimes a)=\lambda a\otimes b+\lambda b\otimes a, &&
c(b\otimes b)=\lambda a\otimes a+\lambda b\otimes b.
\end{align*}

There are two relations in degree two:
\[
	ab=iba,\quad
	ia^2=b^2.
\]
Then $\dim\NA(V,c)=5$. The Hilbert series is 
\[
    H(t)=1+2+2t^2.
\]
A linear basis for $\NA(V,c)$ is $\{1,a,b,a^2,ab\}$. 
\end{example}

%\begin{example}[Rowell]
%\end{example}
%
%
%\begin{block}
%	A braided vector space yields a special type of (braided) Hopf algebra called
%	the \textbf{Nichols algebra} of $(V,c)$. To define Nichols algebras we need the Artin braid 
%	group $\B_{n}$. This group can be presented with generators 
%	$\sigma_{1},...,\sigma_{n-1}$ and relations
%	\begin{align*}
%		&\sigma_{i}\sigma_{i+1}\sigma_{i}=\sigma_{i+1}\sigma_{i}\sigma_{i+1} && \text{for }1\leq i\leq n-2,\\
%		&\sigma_{i}\sigma_{j}=\sigma_{j}\sigma_{i} && \text{for \ensuremath{|i-j|>1}.}
%	\end{align*}
%	Recall that the symmetric group $\Sym_{n}$ can be presented with generators 
%	$\tau_{1},...,\tau_{n-1}$ and relations
%	\begin{align*}
%		&\tau_{i}\tau_{i+1}\tau_{i}=\tau_{i+1}\tau_{i}\tau_{i+1} &  & \text{for }1\leq i\leq n-2,\\
%		&\tau_{i}\tau_{j}=\tau_{j}\tau_{i} &  & \text{for \ensuremath{|i-j|>1},}\\
%		&\tau_{i}^{2}=1 &  & \text{for }1\leq i\leq n-1.
%	\end{align*}
%	Hence there exists a surjection $\B_{n}\to\Sym_{n}$ defined by
%	$\sigma_{i}\mapsto\tau_{i}$. 
%
%	\begin{lem*}[Matsumoto]
%		There exists a set-theoretical section $\mu:\mathbb{S}_{n}\to\mathbb{B}_{n}$,
%		$\tau_{i}\mapsto\sigma_{i}$, such that $\mu(xy)=\mu(x)\mu(y)$ if
%		$\mathrm{length}(xy)=\mathrm{length}(x)+\mathrm{length}(y)$. %The map $\mu$ is
%		%called the \textbf{Matsumoto section}.
%	\end{lem*}
%
%	\begin{rem*}
%		Let $(V,c)$ be a braided vector space, and $n\in\N$. Define
%		\[
%		c_{i}=\id_{V^{\otimes(i-1)}}\otimes\, c\otimes\id_{V^{\otimes(n-i-1)}}\in\Aut(V^{\otimes n})
%		\]
%		for $1\leq i\leq n-1$, i.e.,
%		\[
%		c_{i}\cdot(v_{1}\otimes\cdots\otimes v_{n})=v_{1}\otimes\cdots\otimes v_{i-1}\otimes\, c(v_{i}\otimes v_{i+1})\otimes v_{i+2}\otimes\cdots\otimes v_{n}.
%		\]
%		The operators $c_{i}$ ($1\leq i\leq n-1$) satisfy the defining-relations of the
%		braid group and hence $\rho_{n}:\B_{n}\to\Aut(V^{\otimes n})$,
%		defined by $\rho_{n}(\sigma_{i})=c_{i}$, is a representation of
%		$\B_{n}$ into $V^{\otimes n}$.
%	\end{rem*}
%\end{block}

\section{Problems}

\begin{prob}

\end{prob}

The following problems are open:

\begin{prob}
	Let $\lambda\in\C\setminus\{-1,1\}$ a complex root of $1$. Let $V$ be a
	vector space with basis $\{x_1,x_2,x_3\}$ and $c\colon V\otimes V\to
	V\otimes V$ be given by $c(x_i\otimes x_j)=\lambda x_{2i-j\bmod 3}\otimes
	x_i$.  Is $\NA(V,c)$ finite-dimensional?
\end{prob}

\begin{prob}
	Let $V$ be a vector space with basis $\{x_1,\dots,x_5\}$ and $c\colon
	V\otimes V\to V\otimes V$ be given by $c(x_i\otimes x_j)=-x_{2i-j\bmod
	5}\otimes x_i$.  Is $\NA(V,c)$ finite-dimensional?
\end{prob}

\begin{prob}
	Let $\lambda$ be a primitive third root of one.  Let
	$X=\{(123),(243),(134),(142)\}$ be the conjugacy class of $(123)$ in the
	alternating group $\Alt_4$, $V$ be the vector space with basis $\{x:x\in
	X\}$ and 
	\[
	c\colon V\otimes V\to V\otimes V,\quad
	c(x\otimes y)=\lambda xyx^{-1}\otimes x.
	\]
	Is $\NA(V,c)$ finite-dimensional?
\end{prob}

